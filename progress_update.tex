\documentclass[journal]{IEEEtran}

\usepackage{cite}
\usepackage{graphicx}
\usepackage{amsmath}
\usepackage{algorithmic}

\begin{document}
\title{Political Districting Progress Update}
\author{~David~Fu,~David~Zhang,~Ellen~Wang,~Leslie~Reyes,~Seikun~Kambashi,~Shiranka~Miskin}
\maketitle

% \begin{abstract}
% The abstract goes here.
% \end{abstract}

% \section{Introduction}
% \IEEEPARstart{T}{his} demo file is intended to serve as a ``starter file''
% for IEEE journal papers produced under \LaTeX\ using
% IEEEtran.cls version 1.8b and later.
% I wish you the best of success.


\section{Introduction}

Particle Swarm Optimization is an optimization technique which models particles
moving through a continuous n-dimensional search space, where each particle is a
potential solution to the problem.  At every iteration each particle computes
its velocity via the following formula (from the PSO slides) and adds it to its
current position:

\begin{equation}
\label{vel_update_eqn}
V_{t+ 1} = w \cdot v_t + c_1 r_1 (pbest_t - x_t) + c_2 r_2 (gbest_t - x_t)
\end{equation}

In our report, we propose a novel method of applying Particle Swarm Optimization
to the discrete problem of Electoral Districting.  Electoral districting
problems involve assigning small geographical units (GUs) to a fixed number of
districts such that every district remains contiguous, and every GU is assigned
a district.  Our measure of fitness factors in district compactness and
population equality.

\section{Experiment Design}

Our method is inspired by work conducted by Kang-Ping Wang et al\cite{tsp-pso} on the
TSP\@. In order to apply formula~(\ref{vel_update_eqn}), we must define several
encodings and operations.

\subsection{Solution Encoding}
We define a particle as a set of districts $P = \{D_1, D_2, \dots, D_n\}$ where
each district $D_i$ is a set GUs assigned to that district.  In order to ensure
that district $D_i$ will represent the same general area between all candidate
solutions, we select $n$ GUs which are far apart from each other to function as
``centers'', similar to work done by Ricca et al\cite{voronoi}.  These central
GUs remain constant over all solutions, and district $D_i$ is defined as the
district containing the $i\text{'th}$ center.

\subsection{Velocity Encoding}
We define a Swap Operator $SO(i, D_j)$ as taking GU $i$ belonging to district
$D_c, c \neq j$ and assigning it to $D_j$.  A swap is only valid if $i$ is a
neighbouring unit to district $D_j$.  We now define Velocity to be an ordered
sequence of one or more valid Swap Operators.

\begin{equation}
\label{vel_encoding}
    V = (SO_1, SO_2, \dots, SO_m)
\end{equation}

It is important to note that $SO_{j = i + 1}$ must be valid under the state
resulting from applying $SO_i$.  The removal of any intermediate $SO$ may
invalidate the entire sequence.

\subsection{Addition of Velocities to Solutions}
Given a solution $X$ and a velocity $V = (SO_1 \dots SO_m)$, $X + V$ is the
solution resulting from the application of swaps $SO_1 \dots SO_m$ on $X$ in
ascending order.

\subsection{Subtraction of Solutions from Solutions}
Given solutions $A$ and $B$, the subtraction operator $A - B$ is a
velocity described by a minimal sequence of swaps required to transform solution
$B$ into $A$.  This is determined by iterating through all districts $D_{j}$ in
$B$ and applying $SO(i, D_j)$ if GU $i$ is adjacent to $D_j$ in $B$ and GU $i$ is
assigned to $D_j$ in $A$.

\subsection{Multiplication of Real Numbers and Velocities}
Given $V = (SO_1 \dots SO_m)$, we define $c \cdot V, c \in [0, 1]$ as the
sequence constructed from the first $\lceil c \cdot m \rceil$ swap operators in $V$.

\subsection{Addition of Velocities to Velocities}
Due to a given Swap Operator only being valid on certain solutions, we cannot
simply append two sequences of swaps.  In PSO, velocities are added in order to
have a solution be able to tend toward either the current direction of movement,
the local best solution, or the global best solution.  To achieve this under our
constraints, in order to determine $X_2 = X_1 + c\cdot V + r_1 \cdot d \cdot (A
- X_1) + r_2 \cdot e \cdot (B - X_1)$, we approximate this effect by utilising
the property that $A - B$ will always produce a sequence of swaps which is valid
on solution $B$.

\begin{equation}
\label{vel_encoding}
    \begin{aligned}
        D &= X_1 + c \cdot V\\
        E &= D + r_1 \cdot d \cdot (A - D)\\
        X_2 &= E + r_2 \cdot e \cdot (B - E)\\
    \end{aligned}
\end{equation}

The relationship between coefficients $c$, $d$, and $e$ will need to be tuned to fit this new method of
``addition''.

\section{Software Tools \& Libraries}
In order to conduct simple tests we created our own data set, and for the final
report we have acquired population and shape data for all US states from the
US Census Bureau.

\section{Initial Results}
We developed an initial version of our PSO algorithm as well as an SA
solution based on work by Burcin~Bozkaya et al\cite{local-search}.  The
results show our PSO implementation to be slightly worse, however further
improvements are still possible.  We plan to also implement a Tabu Search
solution, and compare the three approaches using census data for 3 states.

% \subsection{Subsection Heading Here}
% Subsection text here.

% \subsubsection{Subsubsection Heading Here}
% Subsubsection text here.

% \appendices
% \section{Proof of the First Zonklar Equation}
% Appendix one text goes here.

% \section{}
% Appendix two text goes here.

% % use section* for acknowledgment
% \section*{Acknowledgment}

% The authors would like to thank...



\begin{thebibliography}{1}

\bibitem{tsp-pso}
Kang-Ping~Wang et al. Particle Swarm Optimization for Travelling Salesman
        Problem, 2003

\bibitem{voronoi}
Federica~Ricca et al. Weighted Voronoi region algorithms for political
        districting, 2008

\bibitem{local-search}
    Burcin~Bozkaya et al.  A tabu search heuristic and adaptive memory procedure
        for political districting, 2003
\end{thebibliography}
\end{document}


