\documentclass[journal]{IEEEtran}

\usepackage{cite}
\usepackage{graphicx}
\usepackage{amsmath}
\usepackage{algorithmic}

\begin{document}
\title{Political Districting via Discrete Particle Swarm Optimization}
\author{~David~Fu,~David~Zhang,~Ellen~Wang,~Leslie~Reyes,~Seikun~Kambashi,~Shiranka~Miskin}
\maketitle

\begin{abstract}
The abstract goes here.
\end{abstract}

\section{Introduction}
% The Summary should be a brief version of the full report. It should give the
% reader an accurate overview. Be brief, but be specific. 
\IEEEPARstart{T}{his} demo file is intended to serve as a ``starter file''
for IEEE journal papers produced under \LaTeX\ using
IEEEtran.cls version 1.8b and later.
I wish you the best of success.


\section{Literature Review}
% Conduct a critical survey on similar solutions and explain how your solution
% extends or differs from these solutions. 

\section{Problem Formulation and Modeling}
% Include the problem statement and describe its model.

\section{Proposed Solution}


\section{Performance Evaluation}

\section{Conclusions \& Recommendations}


% \subsection{Subsection Heading Here}
% Subsection text here.

% \subsubsection{Subsubsection Heading Here}
% Subsubsection text here.

% \appendices
% \section{Proof of the First Zonklar Equation}
% Appendix one text goes here.

% \section{}
% Appendix two text goes here.

% % use section* for acknowledgment
% \section*{Acknowledgment}

% The authors would like to thank...



\begin{thebibliography}{1}

\bibitem{tsp-pso}
Kang-Ping~Wang et al. Particle Swarm Optimization for Travelling Salesman
        Problem, 2003

\bibitem{voronoi}
Federica~Ricca et al. Weighted Voronoi region algorithms for political
        districting, 2008

\bibitem{local-search}
    Burcin~Bozkaya et al.  A tabu search heuristic and adaptive memory procedure
        for political districting, 2003
\end{thebibliography}
\end{document}


