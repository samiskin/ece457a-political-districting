\documentclass[journal]{IEEEtran}

\usepackage{cite}
\usepackage{graphicx}
\usepackage{amsmath}
\usepackage{algorithmic}

\begin{document}
\title{Political Districting via Discrete Particle Swarm Optimization}
\author{~David~Fu,~David~Zhang,~Ellen~Wang,~Leslie~Reyes,~Seikun~Kambashi,~Shiranka~Miskin}
\maketitle

\begin{abstract}
% The Summary should be a brief version of the full report. It should give the
% reader an accurate overview. Be brief, but be specific. 
The abstract goes here.
\end{abstract}

\section{Introduction}
% Summarize the importance of the problem you are trying to solve and the reason
% that motivated you to select this project. Explain what was the problem or
% challenge that you were given? state the purpose of the project and how did
% you solve it? Enumerate the objectives of the project and describe in brief
% the structure of the report. 
\IEEEPARstart{T}{his} demo file is intended to serve as a ``starter file''
for IEEE journal papers produced under \LaTeX\ using
IEEEtran.cls version 1.8b and later.
I wish you the best of success.


\section{Literature Review}
% Conduct a critical survey on similar solutions and explain how your solution
% extends or differs from these solutions. 

\section{Problem Formulation and Modeling}
% Include the problem statement and describe its model.

\section{Proposed Solution}
% Generate an initial solution.
% Suggest a cost function (objective function) suitable for this problem.
% Define a suitable neighborhood operator.
% Define a suitable solving strategy for this problem.
% Select your own values for the parameter and explain the basis for your selection.
% Describe how each algorithm (TS, SA, GA, PSO and ACO) will proceed to solve this problem by performing at least two hand iterations on a reduced version of the problem.
% Implement the proposed solution using Matlab/Octave/Python.



\section{Performance Evaluation}
% Establish a set of evaluation metrics and run some experiments with different
% values of algorithm parameters to quantitatively and qualitatively assess the
% performance of the developed solution using different meta-heuristic
% optimization techniques. Students must identify the pros and cons of each
% technique and assess the quality of work as well as its fit with project
% objectives.

\section{Conclusions \& Recommendations}
% Summarize the conclusion and future improvement. Explain how did you solve the
% problem, what problems were met? what did the results show? And how to refine
% the proposed solution? You may organize ideas using lists or numbered points,
% if appropriate, but avoid making your report into a check-list or a series of
% encrypted notes 



\begin{thebibliography}{1}
% Every report needs references; in fact, your failure to consult
% references for guidance may be considered negligence. On the other
% hand, when you include sentences, photos, drawings or figures from
% other sources in your report, the complete reference must be cited.
% Failure to do so is plagiarism, an academic infraction with serious
% consequences.

\bibitem{tsp-pso}
Kang-Ping~Wang et al. Particle Swarm Optimization for Travelling Salesman
        Problem, 2003

\bibitem{voronoi}
Federica~Ricca et al. Weighted Voronoi region algorithms for political
        districting, 2008

\bibitem{local-search}
    Burcin~Bozkaya et al.  A tabu search heuristic and adaptive memory procedure
        for political districting, 2003
\end{thebibliography}
\end{document}


